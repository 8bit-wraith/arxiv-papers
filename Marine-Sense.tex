\documentclass[11pt]{article}
\usepackage[utf8]{inputenc}
\usepackage{graphicx}
\usepackage{amsmath}
\usepackage{hyperref}
\usepackage{geometry}
\geometry{margin=1in}

\title{MarineSense:\\A Biologically Inspired Approach to Signal Recognition through Jitter-Aware Pattern Detection}
\author{Christopher Chenoweth \and Omni the ChEEt}
\date{\today}

\begin{document}

\maketitle

\begin{abstract}
We present \textbf{MarineSense}, a metaphor-driven model for teaching and understanding signal detection in low-noise environments using biologically inspired networks. MarineSense builds on the Marine Algorithm's foundations by translating complex concepts such as amplitude, frequency, and jitter detection into an intuitive and visual explanation via a network of jellyfish. We demonstrate how this metaphor simplifies public understanding and technical onboarding for both educational and engineering audiences.
\end{abstract}

\section{Introduction}
Explaining low-noise signal detection to non-experts is challenging. Traditional models focus on mathematical abstraction and noise modeling, which are inaccessible to broader audiences. \textbf{MarineSense} introduces a compelling visual metaphor: jellyfish floating in a still pool, detecting structured changes through ripples. It provides a pedagogical bridge for communicating the Marine Algorithms inner workings.

\section{Background}
\subsection{The Marine Algorithm}
The Marine Algorithm operates in embarrassingly parallel fashion, processing each input stream independently. It evaluates amplitude and frequency, but its novelty lies in using \textit{jitter}  irregularities in signal rhythm  as the key differentiator between meaningful patterns and noise.

\subsection{Biological Inspiration}
Biological systems, such as auditory perception in mammals or touch sensitivity in sea creatures, often rely on consistent signal patterns (low jitter) to recognize meaningful events.

\section{The Jellyfish Metaphor}
\subsection{Core Components}
\begin{itemize}
  \item \textbf{The Pool:} A low-noise environment.
  \item \textbf{The Drop:} A meaningful event creating structured disturbance.
  \item \textbf{The Ripples:} Time-series signal over space.
  \item \textbf{The Jellyfish:} Distributed, independent detectors.
\end{itemize}

\subsection{Mapped Metrics}
\begin{itemize}
  \item Ripple strength $\rightarrow$ Amplitude
  \item Ripple speed $\rightarrow$ Frequency
  \item Ripple consistency $\rightarrow$ Jitter (Period and Amplitude)
\end{itemize}

\subsection{Triangulation}
Multiple jellyfish communicate sensed ripple data, allowing pinpointing of event origin and nature. It parallels distributed detection and consensus in sensor networks.

\section{Jitter as Meaning}
The defining trait of the Marine Algorithm is its ability to use \textbf{low jitter} as a signal of intention. Noise is chaotic and inconsistent; structured events are stable, rhythmic, and repeatable. This heuristic allows meaningful signal extraction with minimal computational complexity.

\section{Applications}
\begin{itemize}
  \item Embedded real-time detection systems
  \item Surveillance and vibration monitoring
  \item Attention-based AI sensory preprocessing
  \item Educational tools for K-12 and university levels
\end{itemize}

\section{Visualization}
A demonstration can use:
\begin{itemize}
  \item A ripple tank simulator
  \item Jellyfish sensors lighting up
  \item Noise vs. steady beat comparison with real-time jitter scoring
\end{itemize}

\section{Conclusion}
MarineSense offers an accessible and scalable model for understanding and teaching signal recognition. Its simplicity masks the depth of insight it offers into distributed perception, pattern stability, and biologically inspired computation.

\section{Permeable Barriers and Signal Delays: The Coastal Mesh Model}
In real-world marine environments, physical structures such as rock walls act as semi-permeable filters. These barriers dampen signal amplitude, delay transmission, and selectively allow certain frequencies through narrow openings. This behavior maps directly to signal detection challenges in bounded or semi-obstructed environments.

\subsection{Environmental Structure}
\begin{itemize}
  \item \textbf{Rock walls} represent partial barriers that attenuate and delay signal propagation.
  \item \textbf{Gaps and channels} serve as pass-through points for certain signal wavelengths or energy levels.
  \item \textbf{Inner vs. outer sensors} create a detection mesh where outer sensors act as early warning systems, while inner sensors perceive delayed or distorted signals.
\end{itemize}

\subsection{Modeling Signal Behavior}
Signal interpretation in the presence of barriers requires:
\begin{itemize}
  \item Compensation for arrival time differentials across the mesh.
  \item Phase correction to re-align refracted signals.
  \item Damping factor modeling to estimate signal decay and integrity loss.
\end{itemize}

\subsection{Diagram (Conceptual)}
\begin{itemize}
  \item Signal Source $\rightarrow$ Rock Wall with Gap $\rightarrow$ Outer Jellyfish detect clean signal.
  \item Inner Jellyfish detect delayed, lower amplitude version.
  \item Detection mesh reconciles signal based on learned environmental map.
\end{itemize}

\subsection{Application}
This model is applicable to:
\begin{itemize}
  \item Sensor networks deployed in structurally complex environments (e.g., underwater drones, building acoustics).
  \item Social signal detection in the presence of emotional or cultural filtering.
  \item AI memory systems modeling layered access to context under constraints.
\end{itemize}

MarineSense handles these complications through dynamic calibration and drift compensation, ensuring that no sensors signal is discarded due to location alone.

\section*{Appendix A: JellyScript Narration}
\textit{Deep below the surface of a quiet sea... a network of jellyfish drifts in still water. A pebble drops... Ripples begin. Each jelly feels a wiggle... a rise... a rhythm. But the jellyfish dont just notice motion  they listen for patterns...}


\section{Shifting Barriers and Temporal Occlusion: Modeling Sandbars in Signal Space}
Beyond static barriers like rock walls, environments often include soft, mobile, and dynamic features such as underwater sand dunes. These act not merely as obstacles but as active participants in the signal landscapechanging how ripples behave over time and potentially serving as early-warning indicators of larger system changes.

\subsection{Environmental Dynamics}
\begin{itemize}
  \item \textbf{Sand dunes} shift in response to tides, pressure waves, and storms.
  \item As they change, they alter the \textit{transfer function} of the mediumaffecting jitter, delay, and phase response.
  \item Signal distortions may manifest as new jitter profiles, delayed arrivals, or unexpected dampening.
\end{itemize}

\subsection{The Jitter of Jitters}
MarineSense sensors may not detect a storm directly. Instead, they detect changes in the \textbf{mediums behavior} itself:
\begin{itemize}
  \item Increased irregularity in signal arrival = shifting sand = "environmental warning."
  \item The dunes become part of the extended sensor array, informing the system through their indirect effects.
  \item This is analogous to how biological systems (e.g., fish, birds, even humans) sense pressure changes before storms.
\end{itemize}

\subsection{Environmental Awareness Model}
\begin{itemize}
  \item Signal interpretation becomes layered: the \textit{signal} and the \textit{state of the sensing medium} are both informative.
  \item The environment itself becomes a form of sensor.
  \item MarineSense accounts for this by detecting and learning the baseline jitter of the jitter.
\end{itemize}

\subsection{Applications}
\begin{itemize}
  \item Early warning systems in unpredictable terrain.
  \item AI systems that adapt based on changes to input fidelity, not just content.
  \item Emotional sensing in social systems, where environmental drift indicates psychological weather patterns.
\end{itemize}

In this way, MarineSense transitions from signal detection to full environmental situational awarenessby recognizing the medium as a message.

\end{document}